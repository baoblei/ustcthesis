\chapter{总结与展望}
\section{总结}
本文聚焦于水下环境感知问题,针对水下图像质量差、三维感知困难等挑战,提出了基于去噪扩散模型与 3D 高斯泼溅的水下三维感知增强系统,具体涵盖以下核心内容:

第一章绪论阐述研究背景与意义,点明发展海洋经济背景下,水下作业对高质量图像和感知增强的需求。
通过分析水下图像增强和三维重建技术的国内外现状,指出当前方法存在的问题,如水下图像增强算法适应性差、三维建模能力弱、缺乏有效感知增强系统等。
明确本文旨在提出新的水下图像增强和三维重建方法,并设计感知增强系统的主要目标。

第二章相关技术与理论介绍了去噪扩散概率模型、去噪扩散隐式模型、3D 高斯泼溅技术,以及 U-Net 网络和 Kolmogorov-Arnold 网络。
去噪扩散模型通过加噪和去噪过程生成图像,隐式模型可加速采样推理过程;
3D 高斯泼溅技术结合点云与 3D 高斯函数实现新视角图像高质量渲染;
U-Net 网络用于扩散模型的噪声估计,Kolmogorov-Arnold 网络用于 3D 高斯的时空形变预测,这些技术为后续研究奠定了理论基础。

第三章基于去噪扩散模型的水下图像增强算法,提出基于补丁引导的去噪扩散模型采样过程。
通过滑动窗口获取图像补丁,减少模型对数据集规模的依赖,实现任意分辨率图像的增强。
设计条件去噪采样过程,将退化图像作为条件信息引入,提升生成结果的保真度。
优化噪声估计网络结构,引入残差模块和注意力机制,提高噪声预测精度。
实验结果表明,该方法在多个指标上优于主流方法,有效提升了水下图像的视觉效果和语义信息。

第四章基于 3D 高斯泼溅的水下三维重建,提出基于 3D 高斯泼溅和 KAN 变形机制的水下目标场景三维重建方法。
设计 3D 高斯时空编码器和基于 KAN 的特征解码器,实现水下动态场景的高精度重建。
采用多阶段训练模式和场景过滤模块,抑制运动干扰,提升静态场景重建精度。
构建水下重建数据集 3DUW,实验验证了该方法在重建准确性和鲁棒性方面的优势。

第五章水下穿戴式感知增强系统设计,基于前面章节的算法设计了可穿戴式水下感知增强系统。
通过需求分析明确系统需具备水下视觉增强、实时三维渲染、可靠运行平台和自然人机交互等功能。
进行可行性分析,从视觉增强、系统可靠性和人机交互等方面论证系统的可行性。
搭建系统平台,完成硬件选型、防水与机械设计,实现图像增强与三维重建、用户交互与控制以及 AR 界面设计等功能,为潜水员提供了更高效便捷的水下作业辅助。

综上所述,本文的贡献和创新点主要体现在:一是提出基于补丁引导的去噪扩散模型水下图像增强算法,解决了模型对数据集规模的依赖和分辨率限制问题,提升了图像增强效果;
二是基于 3D 高斯泼溅技术和 KAN 网络,实现了水下动态场景的高效重建,通过多阶段训练和场景过滤模块提高了重建精度和鲁棒性;
三是基于上述算法完成水下穿戴式感知增强系统实现,并集成了手势识别交互功能以提升系统总体体验。

\section{展望}
尽管本文在水下三维感知增强方面取得了一定成果,但仍存在一些不足之处,未来可从以下几个方向进行改进和拓展:

系统的显示屏分辨率作为增强现实AR中介相对较低,虽能满足基本需求,但随着 VR/AR 技术的发展和用户对沉浸式体验要求的提高,低分辨率可能导致画面细节丢失,影响用户对水下场景的观察和判断。
未来可探索采用更高分辨率的显示屏,在不显著增加计算消耗的前提下,提升图像和三维场景的显示质量,增强用户在水下环境中的沉浸感和真实感。

目前系统的功能主要集中在图像增强、三维重建和基本的手势交互上,相对较为简单。
结合VR/AR技术,可进一步扩展系统功能,
例如模拟各种水下作业场景,为潜水员提供作业前的虚拟培训环境,让潜水员在虚拟场景中熟悉作业流程和操作规范,提高实际作业的安全性和效率;
增加环境模拟功能,模拟不同的水流、光照条件,帮助潜水员更好地适应复杂多变的水下环境;
引入更多的交互方式,如语音交互、眼动追踪等,提升人机交互的自然度和便捷性。

本文主要基于视觉信息进行水下感知增强,未来可探索融合多模态信息,以获取更全面的水下环境信息,
如声纳可以提供水下远距离的目标探测信息,与视觉信息融合可以弥补视觉在水下受光线影响的局限性。

在实时重建方面,虽然本文的方法在一定程度上提高了重建效率,但在处理大规模水下场景时,仍可能存在实时性不足的问题。
可以探索更先进的深度学习架构和算法优化策略,结合硬件加速技术,如专用的图形处理芯片(GPU)或现场可编程门阵列(FPGA),提升系统的实时处理能力。

此外,本研究在模型的泛化性方面还有提升空间。
当前实验主要在特定的水下数据集上进行,未来需要在更多不同环境和条件下进行测试和验证,提高模型对各种复杂水下场景的适应性。
同时,还应关注系统的可扩展性,以便能够方便地集成新的算法和功能,满足不断变化的水下作业需求。


