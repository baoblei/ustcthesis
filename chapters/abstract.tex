% !TeX root = ../main.tex

\ustcsetup{
  keywords  = {水下三维感知, 去噪扩散模型, 3D 高斯泼溅},
  keywords* = {underwater perceptual, denoising diffusion models, 3D Gaussian splatting},
}

\begin{abstract}
  依靠视觉感知水下环境是潜水员进行各种勘探任务时的重要手段,但是由于水介质对不同波长的光存在明显的吸收差异,尤其是对红光的吸收较为显著,导致水下图像常呈现出偏绿色调。同时,水下环境中的悬浮颗粒会在光线传播中产生散射,使得图像出现模糊,严重影响水下成像质量,恶劣的水下图像对各种水下任务如海洋环境观测、海洋工程项目维修、海洋资源探测以及海底遗迹测绘等带来困扰。
  因此,增强潜水员在水下作业时高质量全方位的辅助感知能力成为当前海洋科学领域亟需解决的问题。

  近年来,去噪扩散概率模型和3D高斯泼溅技术在图像生成和三维重建领域取得了显著的进展。
  本文以这两种技术为基础,提出一种基于去噪扩散模型的水下图像增强算法与水下高质量三维场景重建方法,
  并结合水下可穿戴式设备,设计了一种水下三维感知增强系统,旨在为潜水员提供一个可以自由控制视角的三维场景预览,
  潜水员可以利用该系统在潜水过程中查看目标物体的三维信息,从而辅助潜水员进行水下任务。本文具体研究内容如下:

  (1)考虑水下图像存在的严重退化问题,本文提出一种基于去噪扩散模型的水下图像增强算法。
  为了缓解扩散模型训练数据的规模依赖问题以及突破模型输出的尺寸限制,本文提出了基于补丁引导的去噪扩散模型采样过程。
  针对不同分辨率图像输入按照统一大小的补丁对图像进行分解,通过分解后的图像补丁级语义内容预测图像局部区域的噪声,
  将分解后的预测结果进行重新组合形成完整图像的噪声水平,进而采样出每个去噪阶段的完整图像语义信息与重建特征,
  让模型可以对任意分辨率的图像进行处理并生成相同尺寸的输出图像,同时缓解扩散模型训练数据的规模依赖。
  
  (2)本文提出一种基于3D高斯泼溅的水下目标场景高质量重建方法,
  首先将3D高斯在时间维度上进行水下动态场景的建模,结合高效的时空编码器与多头解码器进行3D高斯的时空变形预测,实现水下动态场景的初步重建。
  然后,本文受水流和光照影响的水下动态环境变换存在一个高频分布表示的启发,利用设计的场景过滤模块对学习到的动态重建场景信息进行低频场景数据的过滤,保留水下无扰动的目标场景信息,
  结合三阶段的自对抗训练模式适应水下环境的动态变换,消除水下运动干扰,更好的优化水下场景的高质量静态模式渲染。

  (3)基于上述两种算法,本文设计了一种水下可穿戴式三维感知增强系统。基于虚拟显示技术,本文搭建了一套可适用于水下环境中的可穿戴式系统,
  并结合预训练的水下3D高斯模型为潜水员提供一个可以自由控制视角的三维场景预览画面,
  同时提出一种基于手势识别的人机交互方法,潜水员可以利用手势控制视角的移动,从而获取目标物体的三维信息,辅助潜水员进行水下任务。

  % (1)通过研究去噪扩散模型的图像生成过程,结合水下图像衰减原理,探索利用去噪扩散模型进行水下图像增强的新颖方法,充分发挥扩散模型图像编辑能力恢复图像原始对比度和色彩,减轻由于水下恶劣环境对成像结果带来的语义干扰,为水下更多基于语义的下游任务提供高质量的图像输入基础。

  % (2)为了解决端到端的模型结构对于输入输入尺寸的限制,本文提出一种基于补丁引导的去噪扩散模型采样过程,针对不同分辨率图像输入可以按照统一大小的补丁对图像进行有重叠的分解,通过分解后的图像补丁级语义内容预测图像局部区域的噪声,将分解后的预测结果进行重新组合形成完整图像的噪声水平,进而采样出每个去噪阶段的完整图像语义信息与重建特征,实现任意分辨率的图像生成,同时缓解扩散模型训练数据的规模依赖。

  % (3)基于提出的水下图像增强方法,提出一种3D高斯泼溅技术在水下动态场景中的高质量重建方法。本文通过基于科尔莫格罗夫-阿诺德网络设计的高斯时空编码器与多头解码器进行水下动态场景重建。另外本文受水流和光照影响的水下动态环境变换存在一个高频分布表示的启发,利用学习到的动态重建场景信息进行低频视角数据的过滤,从而能够自适应水下环境的动态变换,消除水下运动干扰,更好的优化水下场景的高质量静态模式渲染。
\end{abstract}

\begin{abstract*}
Relying on visual perception of the underwater environment is an important means for divers to carry out various exploration tasks.
However, due to the significant absorption differences of light with different wavelengths in water, especially the remarkable absorption of red light, underwater images often exhibit a greenish tint. 
At the same time, suspended particles in the underwater environment cause scattering during light propagation, making the images blurry and seriously affecting the underwater imaging quality. 
Poor quality underwater images pose challenges to various underwater tasks such as marine environment observation, marine engineering project maintenance, marine resource exploration, and underwater archaeological site mapping. 
Therefore, enhancing the high quality and comprehensive auxiliary perception ability of divers during underwater operations has become an urgent problem to be solved in the current field of marine science.

In recent years, denoising diffusion probability models and 3D Gaussian splashing techniques have made significant progress in the fields of image generation and 3D reconstruction. 
Based on these two techniques, this paper proposes an underwater image enhancement algorithm and a high quality 3D scene reconstruction method for underwater scenes, both based on denoising diffusion models. 
Combined with underwater wearable devices, an underwater 3D perceptual enhancement system is designed. 
The aim is to provide divers with a 3D scene preview whose view can be freely controlled. Divers can use this system to view the 3D information of target objects during diving, thus assisting them in underwater tasks. 
The specific research contents of this paper are as follows:

(1) Considering the severe degradation problems of underwater images, this paper proposes an underwater image enhancement algorithm based on a denoising diffusion model. 
To alleviate the scale dependence problem of diffusion model training data and break through the size limitation of the model output, a patch guided denoising diffusion model sampling process is proposed. 
Images of different resolutions are decomposed into patches of a unified size. The noise of local image regions is predicted based on the semantic content of the decomposed image patches. 
The predicted results of the decomposed patches are recombined to form the noise level of the complete image, and then the complete image semantic information and reconstruction features at each denoising stage are sampled. 
This allows the model to process images of any resolution and generate output images of the same size, while alleviating the scale dependence of the diffusion model on training data.

(2) This paper proposes a high quality reconstruction method for underwater target scenes based on 3D Gaussian splashing. 
First, 3D Gaussians are used to model underwater dynamic scenes in the time dimension. Combined with an efficient spatio temporal encoder and a multi-head decoder, spatio-temporal deformation prediction of 3D Gaussians is carried out to achieve the preliminary reconstruction of underwater dynamic scenes. 
Then, inspired by the fact that the dynamic environmental transformation of underwater scenes affected by water flow and lighting has a high frequency distribution representation, a designed scene filtering module is used to filter the low frequency scene data of the learned dynamic reconstruction scene information, retaining the undisturbed target scene information underwater. 
A three stage self-adversarial training mode is combined to adapt to the dynamic transformation of the underwater environment, eliminate underwater motion interference, and better optimize the high quality static mode rendering of underwater scenes.

(3) Based on the above two algorithms, this paper designs an underwater wearable 3D perceptual enhancement system.
Based on virtual display technology, a wearable system suitable for the underwater environment is built.
Combined with a pretrained underwater 3D Gaussian model, a 3D scene preview screen whose view can be freely controlled is provided for divers. 
At the same time, a human computer interaction method based on gesture recognition is proposed. Divers can use gestures to control the movement of the view, thereby obtaining the 3D information of target objects and assisting them in underwater tasks.

\end{abstract*}
