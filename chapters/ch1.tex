% !TeX root = ../main.tex

\chapter{绪论}

\section{研究背景与意义}
党的二十大报告指出:发展海洋经济,保护海洋环境,加快建设海洋强国。近年来,随着社会经济的不断发展,人类的生产活动对于各种资源如石油、土地、天然气以及各种生物资源的需求也不断增加,陆地资源的不断减少使得人们将注意力转移至海洋资源。
海洋蕴藏着丰富多样的生物和矿产资源 \cite{ocean_energy},与陆地相比,人类对于海洋资源的开发仍处于探索阶段,随着各种海洋研究和勘探作业如水下观测、海底电缆或管道维护、水下考古遗址测绘和重建、水下自主导航等需求不断增加,自主潜水器(AUV)和遥控潜水器(ROV)己逐渐被应用在海上作业中 \cite{auv}\cite{rov}。
但有些需要极高灵巧操作的水下任务还是需要依靠人工干预,一些水下可穿戴设备可以在人类进行水下活动时提供必要的帮助 \cite{Xia2022},例如柔性外骨骼可以为髋关节内收提供辅助扭矩来增强人类水下活动 \cite{Xia2023},除了实现运动增强之外,水下图像也是作为认知水下信息的重要手段之一,在海洋勘探领域中一直占据着重要地位。
水下潜艇、水下机器人通过在设备终端配置水下相机来采集水下图像、视频数据,常常用于协助人类执行海洋勘探任务,采集到的数据将会被应用下水下科研领域,进而推动人类对海洋环境的探索与认知,因此高质量的水下图像对促进海洋勘探的发展意义重大。

然而,恶劣的水下环境给水下成像带来了一定的问题。
一方面,水不同程度地吸收不同波长的光,导致水下图像的色彩失真。另一方面,水中的悬浮颗粒会使水下光线产生轻微的角度偏转,导致图像的细节模糊。
失真图像不仅给观察者带来较差的视觉感受,还会严重影响水下机器人等搭载的计算机视觉系统的正常工作,因此,许多研究者提出各种图像增强技术对失真图像进行处理,以提高图像质量,便于后续识别、分割等任务的展开。
但是由于不同水下环境中光的衰减和散射程度不同,虽然图像均呈现出亮度低、 细节模糊等特点,但在偏色和模糊程度方面仍存在差异,难以统一处理,给水下图像增强算法的研究带来了巨大的挑战,目前行之有效的鲁棒性水下增强算法数量稀少,大多数算法只能针对一种或几种水体环境进行图像增强,无法处理复杂且多样的水体环境,这极大的限制了算法的应用范围。

为了克服水下图像光照不足的缺点,水下摄像系统通常会搭配人工光源使用,扩大成像范围,但是该方法仍然存在着如下缺点:

首先,人工光源需要大量的能量供给,这会降低水下成像系统的便携性、灵活性,且不利于长时间连续使用;
其次,与自然光源相比,人工光源通常具有非均匀性,使得水下图像出现中间区域亮度较高,图像整体亮度分布不均匀的现象,影响成像质量。

由于复杂的成像环境,水下成像系统无法从根本上消除散射导致的模糊和噪声影响,因此仅靠先进的成像设备是远远不够的。
随着计算机技术的飞速发展,数字图像处理技术获得了长足的发展,日趋成熟,并且在很多领域(如医学、军事、监控、遥感等)都有着重要的应用。
相比于改善成像系统设备所需的高昂代价,数字图像处理方式更为灵活、简便,易于实现。
因此很多研究人员致力于使用数字图像处理技术来改善水下图像的视觉效果,提升成像质量。
水下图像处理技术也越来越受到人们的关注,在改善水下图像成像质量方面有着重要的应用,具有非常广阔的应用前景。

随着工业界和学术界对水下图像增强领域的关注,越来越多的图像增强算法映入眼帘,极大方便了海洋开发与勘探。但是水下环境复杂多变,高效且鲁棒地增强水下图像是一个极具挑战性的任务,因此本文将针对水下图像的增强问题展开深入的研究。针对环境多变、条件恶劣的水下图像增强现状,亟待解决的问题如下:

(1)模型局限性大
与去雨、去雾问题相比,水下图像的建模更为复杂也更为精细。去雨、去雾建模中忽略的一些因素和近似的一些参数,在水下图像增强问题中需要重新考虑,比如水陆环境不同对透射率的影响。除了复杂的水下图像建模过程,传统方法增强水下图像会采用若干已知先验来加强模型对水下环境的刻画,涉及到调整各个约束项之间权重大小,需要人工协调,时间代价昂贵。基于先验的增强方法的优劣程度取决于模型对真实环境的刻画是否完善,对于繁多复杂的水下环境的适应性是否优良,采用的约束是否精确,但是这需要对任务深刻的理解和高超的数学建模技巧以及大量的时间。

(2)数据获取难度大
基于深度学习的水下图像增强方法依赖大量的数据,解决越复杂的问题,网络需要越大的参数量。数据量越大,深度模型鲁棒性越好。同样与去雨、去雾问题相比, 水下图像增强问题不光在真实环境下获得水下图像和清晰的图像难度巨大,采用合成数据的方式获取图像对数据集也极为困难。现有的依靠物理模型合成水下图像数据集的方式,往往通过简化模型来生成有色偏和雾效应的图片,并不能精确描述水下环境。 图像对采用的干净图像往往是陆上的干净图像,这与水下环境并不类似。因此,图像对数据集的缺失为水下图像增强问题的解决带来了困难。

(3)泛化能力弱
目前基于深度学习的水下图像增强算法仍依赖于指定的训练测试集,对于复杂多变的真实场景的水下图像,即训练集尚未包括的水下环境类型,图像增强的结果并不能让人满意,如何扩展数据集,适配多种复杂环境下的水下图像成为水下图像增强算法和实际应用中的巨大挑战。

另外,在水下三维重建渲染方面,目前主流的水下场景重建包括两个方向,一是以深度传感器如声纳为基础的水下稀疏或稠密点云的构建,另一个方向是基于视觉图像的水下三维重建研究,其中基于新视角生成(NVS)的逼真场景重建具有更强的适用性,尤其是对水下人体穿戴式设备来说,通过可以媲美现实观察的图像以及自然实时渲染可以大大提升人类在水下的感知能力。
由于水下环境中存在的图像退化以及水流带来的场景动态扰动,使得水下环境除了采集的图像质量差外,也难以保持一个完全静态的场景,实际进行NVS时要考虑场景的动态影响,包括水流带来的运动变化和光照不均匀带来的颜色变化,但是目前没有真正针对水下这种特殊场景进行去除扰动的重建研究。

\section{国内外研究与发展现状}
\subsection{水下图像增强技术国内外现状}
研究早期,水下图像的增强依赖于多张图像或专用硬件设备,增强过程耗时耗力,且无法支持实时任务。
当前的水下图像增强方法大多基于单张图像且不需要额外的辅助设备,大致可以分为以下三类:基于物理模型的方法、基于先验知识的方法和基于深度学习的方法。
\subsubsection{基于物理模型的方法}
基于物理模型的方法依赖于水下图像生成过程的数学模型来增强水下图像 \cite{uw_img_math1}\cite{uw_img_math2}:
\begin{equation}
I=J \cdot e^{-\beta d}+B^{\infty} \cdot\left(1-e^{-\beta d}\right) 
\label{IFM1}
\end{equation}

其中$I$是水下退化后的图像,$J$是未衰减的图像,$B^{\infty}$是大气光照, $\beta$是衰减系数,$d$是相机到物体的距离。
该模型与陆地图像恢复方法中常用的图像生成过程模型类似 \cite{ifm},用数学模型从退化原理出发表示了水下图像的衰减与散射过程。
2018年,Akkaynak等人\cite{modified_uw_img_math}通过使用$\beta_c^B\left(\mathbf{v}_B\right)$和$\beta_c^D\left(\mathbf{v}_D\right)$修正其中的前向衰减系数和后向衰减系数更新了水下图像生成数学模型:
\begin{equation}
I_c=J_c e^{-\beta_c^D\left(\mathbf{v}_D\right) \cdot z}+B_c^{\infty}\left(1-e^{-\beta_c^B\left(\mathbf{v}_B\right) \cdot z}\right)
\label{IFM2}
\end{equation}

还有一些利用其他信息进行图像恢复的方法,如基于深度图\cite{depth_img}、连续的多帧图像\cite{multi_frame}和偏振图像\cite{U2PNet}等信息的水下图像增强算法。
这些方法可以更好的估计水下图像生成模型中的参数并获得清晰的图像。
多源传感器的信息融合技术\cite{fusion}可以提供更加全面的语义补充来提升水下图像增强的表现,然而它们通常需要特定的硬件支持,如 RGBD 相机和偏振相机。
总的来说,基于物理模型的方法系统地描述了水下图像生成中的存在的问题并为水下图像增强提供了一个数学模型,对水下图像增强任务具有一定的理论指导作用。
但是假设的模型在适应复杂多样的水下环境方面仍然面临挑战,准确估计多个参数仍然是一项具有挑战性的任务。

\subsubsection{基于先验知识的方法}
基于先验知识的水下图像增强方法主要是通过利用对图像亮度、颜色、对比度等方面的已知统计规律来恢复图像质量。
与物理模型方法相比,这类方法通常不需要额外设备支持,仅通过对单张图像的处理即可达到增强效果,因此在实际应用中具有更强的灵活性。
根据先验知识的不同级别可以将基于先验知识的水下图像增强方法分为传统图像处理方法和统计推断方法两大类。

第一种方法涉及应用传统的图像处理技术,根据先验知识将图像的亮度和饱和度等图像属性恢复到正常水平,例如:直方图均衡(HE)\cite{he}、对比度限制自适应直方图均衡(CLAHE)\cite{clahe}、伽马校正和广义反锐化掩蔽 (GUM) \cite{gum},
这类方法不考虑水下图像的成像过程,直接通过直方图拉伸或线性变换等方法对颜色模型中的像素进行修改,以扩展图像的动态像素范围,提高图像对比度并缓解图像色偏问题。
2013 年,Hitam 等人\cite{mixture_clahe}提出一种专门适用于水下图像增强的 CLAHE 方法, 他们在 HSY (由色调、 饱和度和明度构成的色彩空间)和 RGB (由红、绿和蓝构成的色彩空间)模型上进行 CLAHE 运算,并使用欧几里德范数将两个结果组合在一起,有效提高了水下图像的对比度,并降低了图像中的噪声和伪影。
2014 年,Fu 等人\cite{uw_retinex}基于 Retinex 模型的水下图像增强方法,Retinex 模型又称为视网膜大脑皮层理论,该理论基于颜色恒常性,Fu 等人提出的水下图像增强方法采用了 Retinex 的水下图像处理变分框架,他们从水下退化图像中分解出反射分量和照明分量,可以有效恢复图像亮度和校正图像色彩。但这些方法缺乏系统指导,不具备很强的泛化能力。

基于统计推断的先验方法通过统计图像中的潜在规律来估计水下图像生成模型中的相关基本参数,从而实现图像质量提升。
这些方法源于特定环境下的图像先验知识,能够对复杂水下环境进行适应。
何等人 \cite{dcp} 通过观察大量无雾图像的暗通道图数据,发现正常无雾图像的暗通道几乎为0,于是提出了一种暗通道先验(DCP)图像去雾算法,在图像去雾任务中取得了令人惊叹的结果。
受到 DCP 算法在陆地天气去雾为主一系列任务中成功的启发,2013 年,Drews 等人 \cite{udcp}对传统的暗通道先验算法进行了水下的改进,提出一个新的仅考虑绿色和蓝色通道的水下先验方法 UDCP,更加符合水下环境中短波红色光被严重衰减的背景。
然而,由于水下环境复杂,统计出的先验知识可能存在偏差,导致形如DCP这一类的方法在水下可能会表现出亮度或者色彩恢复的不稳定性,甚至在一些极端的水下环境中出现失真或细节丢失的问题。

\subsubsection{基于深度学习的方法}
与传统的基于物理模型和先验知识进行水下图像增强的方法不同,基于深度学习的图像增强方法可以基于神经网络从大规模数据中学习到图像特征和潜在的非线性映射关系。
神经网络模型通过构建复杂的网络结构,可以有效地捕捉图像中的低光、噪声、色彩偏移等特性,并在增强过程中实现对图像细节的高保真还原。
因此,深度学习方法在处理复杂的水下图像时展现出强大的适应性和鲁棒性。
经典的深度学习方法往往以有监督学习的方式,经过多次的卷积和反置卷积操作进行特征提取和恢复来重建清晰的图像。
这些方法在图像处理领域已经展现了显著效果,如图像超分辨率\cite{cyberCSRnet}、图像去雾\cite{QCNN_H}、曝光校正\cite{Exposure_restoration}和语义重建\cite{InpaintingMissingArea}等任务。
有监督的深度学习方法旨在通过对应的特征解码器和编码器以数据驱动的方式学习权重来恢复水下图像,即算法需以包含原始水下图像和其对应理想参考图像的数据集作为训练数据集。
2020 年,Anwar 等人 \cite{uwcnn}提出了 UWCNN,他们引入了一种专门用于水下场景图像增强的卷积神经网络(CNN)模型,并且使用从现实世界地面图像生成的各种合成水下风格图像来训练模型。
然而,由于训练数据的合成性质,这种方法在模型的泛化性上带来了挑战。

水下理想参考图像获取困难,研究人员很难同时获取真实的水下场景和其相应的理想参考图像,因此现有深度学习增强方法多流行使用基于生成对抗网络(GAN) \cite{gan}的方法解决这种问题,GAN 可以在合成数据集的同时实现水下图像的增强。
2018 年, Li 等人 \cite{water_gan}提出了 WaterGAN 方法,该方法以 RGB 图像以及对应的深度图分别作为输入,从而将网络分为水下图像合成和颜色恢复两个部分,图像合成部分用于输出合成水下图像,颜色恢复部分将未经标记的水下图像作为输入,输出恢复后的图像及对应深度图。
同年,Fabbri 等人 \cite{ugan}提出 UGAN,该方法只需输入单张图像,不需要深度图等额外信息的输入,实现了对水下彩色图像的有效增强。
2019 年, Chen 等人 \cite{gan-rs}提出的 GAN-RS 利用多个判别器来引导发生器抑制水下噪声,多个鉴别器可以让GAN-RS 增强其生成器生成更真实且无噪声的水下图像的能力。
2020 年,Islam 等人 \cite{funie_gan}提出了一个快速水下图像增强算法 FUnIE-GAN,并将其应用到了水下机器人的目标检测以及人类姿势检测任务中。作者还利用 CycleGAN \cite{cycle_gan}生成了一个名为 EUVP(Enhanced UnderwaterVision Dataset)的大型水下数据集,其中包含数万张已配对的水下图像。
2021 年,Zhu 等人 \cite{dehaze_gan}提出了 DehazeGAN,这是一种利用多尺度生成器来预测水下图像生成模型中的大气光和传输图的方法,实现通过结合 GAN 与 水下图像生成模型来恢复水下图像。
但由于GAN无法控制生成样本的模式,训练过程经常会遇到收敛困难等问题。同时由于水下场景的复杂性和多样性,研究人员在尝试迁移预先训练的 GAN 模型以适应新环境时经常会出现模型不稳定或失败等问题,因此基于 GAN 的水下图像增强方法在鲁棒性和泛化方面表现不够理想。

与 GAN 相比,去噪扩散概率模型 \cite{pre_ddpm}\cite{ddpm}是一种基于概率似然的模型,且在近年来已经证明了其具有生成高质量图像的能力。
去噪扩散模型以其简单性和可扩展性而闻名,该模型利用一个加噪声的扩散过程从简单的初始噪声分布逐步生成样本,生成的结果相比GAN具有更好的图像的质量和真实性,这使其对包括水下图像增强在内的各种图像生成任务具有吸引力。
2021 年,OpenAI 的研究人员通过实验证明扩散模型比 GAN 在大多是任务中可以表现的更好, 并且他们提出了一种由分类器引导的扩散训练方法\cite{ddpm_beat_gan},确保生成的样本保持其真实性并与所需的目标分布保持一致。

然而,去噪扩散模型的实时性较差,特别是在处理高分辨率图像的生成时需要较长的时间。
2022 年,Rombach 等人 \cite{latent}借助自动编码器,提出一种在图像潜在空间进行扩散训练的方法,以在复杂性降低和细节保留之间实现近乎最优的权衡。
另一种缓解扩散模型运行时间问题的方法是缩短扩散模型的反向采样步骤,如改进的隐式方法 \cite{improved_ddpm,ddim}设计了一种新的采样器,可以用更少的时间步进行去噪采样,从而加速扩散模型的推理过程。
通过利用这一改进,扩散模型中的采样过程变得更加高效,这一进步有助于增强各种应用中扩散模型的整体性能和可扩展性。
2023年,Özdenizci 等人 \cite{weather}利用条件扩散模型在恢复恶劣天气条件下的图像研究中取得了不错效果,他们提出了一种基于补丁的方法来恢复不同尺寸的图像。然而,他们将图像分辨率统一为 16 的倍数,以便于获得补丁,但这仍然引入了图像在像素空间的重新调整,导致图像生成过程中的细节丢失。

总的来说,基于目前去噪扩散模型在图像生成领域表现的相比与GAN的优越性,可以看出去噪扩散模型在水下图像增强方法具有一定的潜力,尽管其生成时间较长,但可以在潜在空间训练或者通过优化采样步骤等方式提升效率和性能。

\subsection{水下三维重建技术国内外现状}
水下三维场景重建技术是AUV、ROV以及水下虚拟现实设备的重要组成部分,其发展对于海洋工程、环境保护和水下探索具有重大意义。
近年来,国内外研究人员围绕该技术的关键问题展开了深入研究,主要集中于以下两个方向。

\subsubsection{传统三维点云重建方法}
传统三维点云重建方法一种是基于深度传感器(如声纳、激光等)等设备,通过采集水下环境的稀疏或稠密点云实现三维建模 \cite{sonar1}\cite{sonar2},
其在海洋勘探、地形测绘和管道维护等工程应用中得到了广泛应用。深度传感器在水下环境中表现出较强的抗干扰能力,但受限于设备成本及数据处理复杂性,其应用仍面临较大挑战。
另外,基于计算机视觉的三维点云重建方法也具有重要地位,代表性技术包括SfM( Structure from Motion)\cite{sfm1, sfm2} 和Multi-view Stereo(MVS) \cite{mvs}。SfM通过分析多视角图像的几何关系生成稀疏点云,而MVS则进一步利用视差信息进行三角测量,生成高精度的稠密点云。
其都需要对来自不同角度的图像进行特征提取和匹配,常用算法如SIFT \cite{sift} 和 ORB \cite{orb},在图像特征点对齐方面具有较高的可靠性。
近年来,机器学习的引入极大地推动了该领域的发展。
例如,2023年Lindenberger等人 \cite{lightglue} 提出了基于神经网络的图像特征匹配方法,该方法通过跨图像对局部特征的学习,大幅降低了内存和计算需求,同时提升了匹配精度。
2024年,Wang等人 \cite{enhancing-mvs} 引入了Twin-FPN架构,结合多尺度特征损失,在复杂的室内场景中实现了更高效的通用特征表示。

其中基于SfM \cite{sfm1, sfm2} 和 MVS \cite{mvs} 是两种经典的点云生成算法,可以通过对物体的多角度拍摄,计算得到物体的稀疏点云和稠密点云。其都需要对不同角度的图像进行特征提取和匹配,通常使用的算法有SIFT \cite{sift}、ORB \cite{orb}等,MVS会进一步利用多幅图像对视差信息进行三角测量,得到物体的三维点云。
2023年,Lindenberger等人 \cite{lightglue}通过使用神经网络学习匹配跨图像对局部特征,能够在较低的内存和计算量下保证较高的匹配精度。
2024年,Wang等人 \cite{enhancing-mvs}引入一种Twin\-FPN架构,结合多尺度特征损失在复杂的室内场景中实现更有效的通义特征表示。
然而,传统三维重建方法在水下应用中受到诸多限制。由于水下环境中光照不均匀、水体吸收和散射造成的图像退化,这些方法在重建精度和效率方面存在明显不足,制约了其在水下场景中的适用性。

\subsubsection{基于图像的新视角生成}
第二类方法是基于视觉图像的水下三维重建 \cite{vision_recon},其中新视角生成(NVS)技术因其能生成逼真的场景表示而备受关注。通过NVS技术,可实现高精度的水下场景渲染,特别是在水下人体穿戴式设备中的应用 \cite{Xia2023},其实时性与逼真性极大地增强了用户的环境感知能力,显著推动了水下虚拟现实技术的发展。
神经辐射场(NeRF) \cite{nerf}\cite{Mip_nerf360} 是NVS领域的代表性框架。
NeRF通过定义隐式函数表征三维场景,并通过体积渲染 \cite{volume_render} 方法生成高质量的多视角图像。
然而,但是由于在光线行进路线上的频繁采样让NeRF具有高额的计算成本,限制了其在实时场景渲染中的应用。
为此,2022年提出的Instant NeRF \cite{instant_nerf} 通过多功能编码器显著降低了NeRF的训练成本,允许在几乎不牺牲质量的情况下利用较小的网络来大大减少浮点运算和内存访问次数。
此外,2023年Cao等人提出了HexPlane方法 \cite{hex_plane},通过高效的融合从每个平面中提取的向量来计算时空点的特征,在动态场景上将训练时间减少了100多倍。

与NeRF的隐式表示不同,3D高斯泼溅(3D Gaussian Splatting) \cite{3DGS} 提供了一种显式的场景表示框架,结合可微分光栅化渲染,显著提升了渲染速度与质量。
基于3D高斯的可变形扩展\cite{dynamic_3DGS1}\cite{dynamic_3DGS2}\cite{dynamic_3DGS_efficient}\cite{4DGS},更是实现了动态场景的实时三维渲染。


2024年,Guo等人\cite{flow_2024motion}为了更好的捕捉场景中的动态信息,通过在训练中增加光流损失对变形范式进行优化,建立了高斯运动和图像像素流之间的对应关系。

\section{本文研究内容与结构安排}
基于上述对水下图像增强任务以及三维重建工作的研究现状与问题分析,本文针对目前水下图像恢复难、水下三维感知质量差、缺乏有效的水下三维场景增强算法等问题,提出了基于去噪扩散模型以及3D高斯泼溅技术的水下图像增强以及三维重建方法,
在此基础上设计完整的一套水下三维增强感知系统,用以水下作业时的辅助视觉感知或者在陆地上的全方位提前观察。
本文的主要研究及结构如图\ref{img:overview}所示。
\begin{figure}
    \centering
    \includegraphics[width=0.8\textwidth]{figures/ch1/overview.pdf}
    \caption{本文主要研究内容}
    \label{img:overview}
\end{figure}

(1)技术前沿性:随着深度学习技术的发展,去噪扩散概率模型在图像生成领域逐步取代生成对抗网络的在许多任务中的应用,本文拟将去噪扩散概率模型和去噪扩散隐式模型应用于水下图像增强,该技术代表了图像生成领域的前沿发展,具有强大的图片编辑能力;同时,3D高斯泼溅技术在三维重建领域取得了显著的进展,本文拟将3D高斯泼溅技术应用于水下三维重建,该技术具有高效的渲染速度和质量,能够实现动态场景的实时表示。

(2)创新处理策略:在水下图像增强方面,通过分解图像为小的重叠补丁并独立处理,对图像局部区域的噪声估计融合完整的图像在不同扩散阶段的噪声,这种做法可以在减轻噪声估计网络整体结构,同时适用于任意分辨率的图像处理;在水下三维重建方面,通过采用4D动态高斯泼溅技术,实现水下动态场景的重建,同时通过自对抗训练消除运动干扰,能够优化水下场景的静态基础表示,为水下三维重建提供新的解决方案。

(3)系统完整性:本文将水下图像增强技术和水下三维重建技术相结合,设计一套完整的水下三维增强感知系统,该系统能够提高水下图像质量和三维场景重建精度,为水下作业提供辅助感知增强,具有较强的实用性和应用前景。

\section{本文的组织结构}
第一章 绪论
本章主要介绍研究背景和意义,提出水下三维感知增强的必要性,详细阐述水下图像增强和三维重建的研究现状,分析现有方法的不足之处以及当前研究面临的挑战,最后明确本文的研究目标和主要内容。

第二章 相关技术与理论
本章分析了去噪扩散模型和3D 高斯泼溅技术的理论知识,同时介绍了Kolmogorov-Arnold 网络(KAN这种新型神经网络模型基本原理,为后文的水下图像增强算法、三维重建框架以及其中的动态点云预测模块提供理论支持。

第三章 基于去噪扩散模型的水下图像增强算法
本章介绍了基于去噪扩散模型的水下图像增强算法,通过补丁引导的采样过程来提高模型的处理效率和图像质量,并设计了一种多阶段网络结构及其训练方法,详细描述了实验数据和评价指标,全面分析了提出的方法对水下图像增强的提升效果。

第四章 基于3D高斯泼溅的水下三维重建
本章详细介绍了基于3D高斯泼溅的水下三维重建方法,通过4D动态高斯泼溅技术实现水下动态场景的实时重建,并设计了自对抗训练方法来优化静态场景的重建效果。实验部分验证了方法在不同水下环境中的重建精度和鲁棒性。

第五章 水下感知增强系统设计
本章首先介绍了有关水下感知平台的外观、防水结构和硬件设计,
搭建了一套包含相机、头戴式显示器、嵌入式处理器和防水背包的平台,
基于第三章与第四章的内容对水下图像增强和三维重建进行功能集成,
形成一套完整的适用于在复杂水下场景下为潜水员提供辅助感知的视觉增强系统。



第六章 结论与展望
本章总结了全文研究成果,回顾基于去噪扩散模型和3D高斯泼溅技术在水下图像增强和三维重建任务中的优势和创新点。同时,对未来的研究方向进行展望,如在复杂水下环境下的多模态融合感知及实时重建的改进,旨在为海洋科学领域的智能感知提供新的技术参考。
